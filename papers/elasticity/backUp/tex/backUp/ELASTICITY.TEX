%% PNAStmpl.tex
%% Template file to use for PNAS articles prepared in LaTeX
%% Version: Apr 14, 2008


%%%%%%%%%%%%%%%%%%%%%%%%%%%%%%
%% BASIC CLASS FILE 
%% PNAStwo for two column articles is called by default.
%% Uncomment PNASone for single column articles. One column class
%% and style files are available upon request from pnas@nas.edu.
%% (uncomment means get rid of the '%' in front of the command)

%\documentclass{pnasone}
\documentclass{pnastwo}

%%%%%%%%%%%%%%%%%%%%%%%%%%%%%%
%% Changing position of text on physical page:
%% Since not all printers position
%% the printed page in the same place on the physical page,
%% you can change the position yourself here, if you need to:

% \advance\voffset -.5in % Minus dimension will raise the printed page on the 
                         %  physical page; positive dimension will lower it.

%% You may set the dimension to the size that you need.

%%%%%%%%%%%%%%%%%%%%%%%%%%%%%%
%% OPTIONAL GRAPHICS STYLE FILE

%% Requires graphics style file (graphicx.sty), used for inserting
%% .eps files into LaTeX articles.
%% Note that inclusion of .eps files is for your reference only;
%% when submitting to PNAS please submit figures separately.

%% Type into the square brackets the name of the driver program 
%% that you are using. If you don't know, try dvips, which is the
%% most common PC driver, or textures for the Mac. These are the options:

% [dvips], [xdvi], [dvipdf], [dvipdfm], [dvipdfmx], [pdftex], [dvipsone],
% [dviwindo], [emtex], [dviwin], [pctexps], [pctexwin], [pctexhp], [pctex32],
% [truetex], [tcidvi], [vtex], [oztex], [textures], [xetex]

%\usepackage[dvips]{graphicx}

%%%%%%%%%%%%%%%%%%%%%%%%%%%%%%
%% OPTIONAL POSTSCRIPT FONT FILES

%% PostScript font files: You may need to edit the PNASoneF.sty
%% or PNAStwoF.sty file to make the font names match those on your system. 
%% Alternatively, you can leave the font style file commands commented out
%% and typeset your article using the default Computer Modern 
%% fonts (recommended). If accepted, your article will be typeset
%% at PNAS using PostScript fonts.


% Choose PNASoneF for one column; PNAStwoF for two column:
%\usepackage{PNASoneF}
%\usepackage{PNAStwoF}

%%%%%%%%%%%%%%%%%%%%%%%%%%%%%%
%% ADDITIONAL OPTIONAL STYLE FILES

%% The AMS math files are commonly used to gain access to useful features
%% like extended math fonts and math commands.

\usepackage{amssymb,amsfonts,amsmath}

%%%%%%%%%%%%%%%%%%%%%%%%%%%%%%
%% OPTIONAL MACRO FILES
%% Insert self-defined macros here.
%% \newcommand definitions are recommended; \def definitions are supported

%\newcommand{\mfrac}[2]{\frac{\displaystyle #1}{\displaystyle #2}}
%\def\s{\sigma}


%%%%%%%%%%%%%%%%%%%%%%%%%%%%%%
%% Don't type in anything in the following section:
%%%%%%%%%%%%
%% For PNAS Only:
\contributor{Submitted to Proceedings
of the National Academy of Sciences of the United States of America}
\url{www.pnas.org/cgi/doi/10.1073/pnas.0709640104}
\copyrightyear{2008}
\issuedate{Issue Date}
\volume{Volume}
\issuenumber{Issue Number}
%%%%%%%%%%%%


\begin{document}

%%%%%%%%%%%%%%%%%%%%%%%%%%%%%%


%% For titles, only capitalize the first letter
\title{Elasticity}

%% Enter authors via the \author command.  
%% Use \affil to define affiliations.
%% (Leave no spaces between author name and \affil command)

%% Note that the \thanks{} command has been disabled in favor of
%% a generic, reserved space for PNAS publication footnotes.

%% \author{<author name>
%% \affil{<number>}{<Institution>}} One number for each institution.
%% The same number should be used for authors that
%% are affiliated with the same institution, after the first time
%% only the number is needed, ie, \affil{number}{text}, \affil{number}{}
%% Then, before last author ...
%% \and
%% \author{<author name>
%% \affil{<number>}{}}

%% For example, assuming Garcia and Sonnery are both affiliated with
%% Universidad de Murcia:
%% \author{Roberta Graff\affil{1}{University of Cambridge, Cambridge,
%% United Kingdom},
%% Javier de Ruiz Garcia\affil{2}{Universidad de Murcia, Bioquimica y Biologia
%% Molecular, Murcia, Spain}, \and Franklin Sonnery\affil{2}{}}

\author{Laurence T. Kell \affil{1}{ICCAT Secretariat, C/Coraz\'{o}n de Mar\'{\i}a, 8. 28002 Madrid, Spain.},
        Finlay Scott     \affil{2}{The Fisheries Laboratory, Lowestoft, NR33 0HT, Suffolk, England.},
        Paul De Bruyn    \affil{3}{AZTI Tecnalia. Herrera kaia portualdea z/g, 20110 Pasaia, Gipuzkoa, Spain},
        Richard D.M. Nash\affil{4}{Institute of Marine Research, PO Box 1870 Nordnes, 5817 Bergen, Norway}
\and
        Mark Dickey-Collas\affil{5}{Wageningen IMARES, P.O. Box 68, 1970 AB IJmuiden, The Netherlands  ICES, 
                                    H. C. Andersens Boulevard 44-46, DK-1553 Copenhagen V, Denmark.}} 

\contributor{Submitted to Proceedings of the National Academy of Sciences of the United States of America}

%% The \maketitle command is necessary to build the title page.
\maketitle

%%%%%%%%%%%%%%%%%%%%%%%%%%%%%%%%%%%%%%%%%%%%%%%%%%%%%%%%%%%%%%%%
\begin{article}

\begin{abstract}
When exploiting populations, references points for exploitation 
rate or biomass are often set as targets or limits. Determining these 
reference points can be difficult, especially when there is little 
understanding of a populations dynamics. A lack of understanding
is often confused with a lack of data on exploitation or trends 
over time of a population. Here we show that elasticity analysis can 
be used to investigate and prioritise which elements of uncertainty 
in life history dynamics are relevant when determining reference 
points for exploited populations. The analysis allowes several important 
questions to be addressed. What is the relative impact of the
different biological processes and parameters on the estimates of 
population status and exploitation? Does the impact depend on the 
reference point and quantity (e.g. SSB or F) chosen or on the status 
of the stock i.e. does knowledge of particular parameters and
processes depend on whether the population is depleted or within 
safe biological limits? Does the exploitation rate impact on the 
suitability of reference points? We suggest that answering these 
questions with the aid of elasticity analysis will help in choosing 
robust target and limit reference points and prioritising research effort.
\end{abstract}


%% When adding keywords, separate each term with a straight line: |
\keywords{term | term | term}


\keywords{Elasticity | fisheries management | life history}

%% Optional for entering abbreviations, separate the abbreviation from
%% its definition with a comma, separate each pair with a semicolon:
%% for example:
%% \abbreviations{SAM, self-assembled monolayer; OTS,
%% octadecyltrichlorosilane}

\abbreviations{MSY, maximum sustainable yield; SRP, spawning reproduction potential}

%% The first letter of the article should be drop cap: \dropcap{}
%\dropcap{I}n this article we study the evolution of ''almost-sharp'' fronts

%% Enter the text of your article beginning here and ending before
%% \begin{acknowledgements}
%% Section head commands for your reference:
%% \section{}
%% \subsection{}
%% \subsubsection{}

\dropcap{T}he adoption of the Precautionary Approach to fisheries management 
\cite{garcia1996precautionary} requires a formal consideration of uncertainty. 
An important principle of the approach is that the level of precaution should 
increase as uncertainty increases so that risk remains constant. So that yields  
are greater in data rich than poor situations. Where the definition of data rich 
and poor is often made on the availablity of fisheries data. 
This can obscure the fact that considerable uncertainty also often exists about 
biological processes such as natural mortality, recruitment and population structure
for commercially important fish populations. For example natural mortality is unknown
for many stocks and is commonly assumed to be constant. However, even when 
data are  limited, empirical studies have shown that life history parameters, such 
natural mortality and as age at first reproduction are strongly correlated with 
and growth rate and size \cite{roff1984evolution}. 

Biological knowledge can therefore be important both for evaluating the robustness 
of advice obtained from data-rich stock assessments and in allowing general rules, 
to be derived and transferable to all populations. For example about choice of 
indicators of stock status, i.e. reference points to indicate whether biomass is too low 
or fishing pressure too high, for use in fisheries management. 

Fisheries management is concerned with setting and then trying to achieve, realistic management 
objectives.  This requires defining limit and target reference points, e.g. for spawning stock 
biomass (SSB) and  fishing mortality (F) and agreeing associated management measures.

However, achieving agreed management objectives is made difficult by, amongst other 
things, the impact of biological and ecological uncertainty on the dynamics of the population.
A key question for fisheries management is therefore how can research be prioritised
so that the impact of biological uncertainty on achieving management objectives
be reduced? In other words, how can we provide advice that is robust to uncertainty?
Answering this question requires the evaluation of the relative importance of the
underlying biological assumptions made  with respect to the
management measures of interest. For example, does uncertainty about recruitment
have a relatively bigger effect on longterm yield and sustainability than uncertainty
about natural mortality? 

Senstivity and elasticity analyses can be used to evaluate the effect of changes in system 
parameters on system outputs. Sensitivities measure absolute changes, for example, by how much does 
the estimate of MSY change as the estimate of age at first maturity change. Elasticities measure 
the relative change and can be used to compare between a range of different sources of uncertainty. 
For example, does the length of first maturity have a larger proportional impact on estimates of MSY 
than the steepness of the stock recruitment relationship? 

Sensitivity analysis is often used in stock assessment to see how much the perception of the stock changes
when an option of a stock assessment program is changed. However, elasticity analysis has seldom been used 
although commonly used in financial and economic managment and despite being applied in conservation biology \cite{de1986elasticity}.
In the latter case it is has been used to relate
changes in vital rates to changes in the population life history \cite{grant2003density} and to quantities
of importance in management such as population viability \cite{heppell1998application}. Previously, 
elasticity anaysis has focused on terrestrial ecology \cite{Benton1999467, Hunter2000299, Pichancourt200631} 
with limited application to marine populations \cite{RogersBennett2006, Heppell2007}. The applicability of 
this approach to resource management has therefore been demonstrated and here we use it to evaluate the 
relative importance of the biological assumptions made in fishery stock assessment that are too seldom questioned.

Consideration of uncertainty within fisheries advice frameworks has been conducted using Bayesian approaches 
or Management Strategy Evaluation (MSE) and is regarded as state-of-the-art. MSE is commonly used to evaluate the 
impact of different management measures, given a broad
range of uncertainty. However, performing an MSE is a costly process in terms of human
resources, can take several years and requires a high level of expertise. Therefore, tools 
such as elasticity analysis, which is comparatively less demanding to carry out, will be important 
to help identify and focus and prioritise research and management efforts. For example, to answer such questions as
is it more important to reduce uncertainty the biology or to develop better management measures such 
as reference points that are robust to such uncertainty?  

Elasticity analyses can also shift the current focus from defining populations either as data poor or
rich defined solely on fishery catch and effort towards a better understanding of
biological processes. Here we demonstrate the use of elasticity analysis for
prioritising research effort with a study of the population dynamics of a fish population
based on life history theory. Ourstudy is not a case study that considers a single species, 
instead we simulate a population based on life history relationships in order to develop
a framework that can be used to derive general rules and applied to a range of
case studies.

Our elasticity analysis evaluates the relative importance of the biological processes
and assumptions for assessing a stock relative to biological reference points. We do this by projecting 
a population from an unfished to an over-exploited state and then evaluate
the elasticity of stock status and fishing mortality relative to a range of reference points.
The elasticities allow us to evaluate the relative importance of
the different biological parameters when assessing the population relative
to system characteristics defined by the reference points.
We use relative estimates since these have been shown to be more robust that absolute estimates \cite{kell2003evaluation} and
are used ny the tuna Regional Fisheries Management Organisations (tRFMOs) when providing advice.
This allows us to address two important questions: what is the relative importance of the different
biological processes in providing advice and how robust is advice based on the
common biological reference points?

\section{Results}

The assumed growth, natural mortality, proportion mature and selectivity-at-age for the simulated stock are shown in Figure 1.
The relationship between the equilibrium values of SSB, yield, recruitment and $F$ are are shown in Figure 2;
the corresponding values for the MSY, $F_{0.1}$ and $F_{crash}$ reference points are also shown. 

The equilibrium dynamics over a range of fishing mortalities from 0 to 75\% of $F_{crash}$ are presented as a phase plot in figure 3,
where the x-axis corresponds to $biomass$ relative to $B_{MSY}$ and the y-axis to $harvest$ relative to $F_{MSY}$.
The quadrants are defined for stock and fishing mortality relative to $B_{MSY}$ and $F_{MSY}$; i.e. red when $B<B_{MSY}$ and $F>F_{MSY}$, green 
if $B$ $\geq$ $B_{MSY}$ and $F$ $\leq$ $F_{MSY}$ and yellow otherwise. The red quadrant therefore refers to an overfished stock subject to overfishing and  
green to a stock which is neither overfished or subject to overfishing.

The results of the elasticity analysis, i.e. SSB and fishing mortality relative to each of the three reference points with respect to the 
biologicalparameters are plotted in figures 4 and 5. The vertical line indicates the boundary between the red and green quadrants and the 
horizontal line where the value of the elasticity is 0, i.e. where varying a parameter has no effect on the measure of interest. 

The plots allow us to idntify the relative importance of the different biological processesand and how robust is advice based on the three reference points?
Figure 4 shows that elasticites of are similar for MSY and $F_{0.1}$, i.e. increasing in magnitude as fishing mortality increases (and SSB declines). Natural mortality
is the parameter with the biggest effect, followed by steepness and $L_{\infty}$. Trends and relative importance by parameter are similar for $F_{Crash}$.
However the magnitudes of the elasticities are not, changing sign as fishing mortality increases. This means that $F_{Crash}$ is robust to chanegs in M when 
the target levels of $B_{MSY}$ is reached and robust to the assummed value of steepness at high fishing mortalities and for a depleted stock.

Inspection by process shows that for growth the key parameter is $L_{\infty}$, for Maturity it is age at first maturity, for M both the average level and 
how fast declines as fish grow are equally important and for selectivity the age at recruitment to the fishery is more important than whether
selectivity declines at older ages.

Figure 5 repeats the analysis for fishing mortality; in this case elasticity is not a function of the level of
fishing mortality or SSB. M is again the most important parameter, followed by steepness and $L_{\infty}$ and parameters have the same relative importance
by process.  
  

\section{Discussion}

%Modern fishery management, particularly in the context of the implementation of the precautionary approach, 
%has become closely associated with harvest management strategy and control rules evolved from control theory. 
%Saila (1997) notes that although “conventional control theory has been a tremendous success where the system 
%is very well defined, such as in missile and space station guidance” the theory is perhaps not appropriate to 
%control complex systems such as commercial fisheries. Amongst the reasons, he mentions that the precise structure 
%of the system is virtually unknown and that there are no reliable models of the process to be controlled. He 
%suggests that success in fishery management requires a system that can handle qualitative information and uncertainty, 
%in a straightforward, transparent and not computationally intensive manner. This is not the direction in which most 
%fishery management processes are heading.

There is a need for a conceptual change from catagorisng stocks as data or information
rich or poor based on the requirements of stock assessment models. An understanding 
of biology and how to provide robust advice given is equally if not more important.
Elasticity analysis can help in identifying where a better understanding is required and
the benifits of obtaining it.	

The collection of information needed to support the science that underpins
advice needs to be prioritised based on management objectives. For example, the definition
of data rich often means data sufficient to conduct aged based assessments such as virtual 
population analysis (VPA). Rather, researchers should ask do we have sufficient knowledge, e.g. about stock structure
and the variability in productivity to ensure that the assumptions of VPA are met
and that the advice based on the analysis is robust.

Fisheries science often focuses on favoured factors, such as catch and time trends in
abundance, and pretends that other processes, such as natural mortality and spatial
structure are unimportant to the sustainable exploitation of the populations.

The approach described here will allow researchers to rationally prioritise what uncertainty
in the assumptions about biological processes impact on the setting of management objectives.
Stock assessment mainly considers uncertainty in observations and processes such
as recruitment variability when uncertainty about the dynamics (i.e. model uncertainty)
has a larger impact on achieving management objectives \cite{punt2008refocusing}. 
Relationships between body size and life history parameters have been described for 
many taxonomic groups 
\cite{blueweiss1978relationships}.
This means that biological parameters are often strongly correlated and so life
history theory can be used to evaluate the assumptions made in stock assessments,
infer parameters for data poor stocks and allow general rules and principles to be
derived. A main aim of this study is to use life histories relationships and elasticity
analysis to evaluate the relative importance of different biological processes and
provide robust advice on the uncertainty about those processes.

This approach will provide insight into what characteristics from population considered
data rich can reliably be transferred to data poor populations.
Currently the length and reproductive traits of populations are being used to
link data rich and data poor populations. However, it has not yet been tested if this is appropriate.

The evaluation of the value of biological information can be done through a variety of approaches,
e.g. developing priors for use in Bayesian estimation or by conducting MSE.
However, such approaches can be relatively complex and time consuming so that they are
unlikely to be routinely applied to many stocks. Applications also tend to be case specific and so difficult
to compare. While sensitivity analyses are often conducted when assessing a stock and in
some fora alternative assessment models when making projections to set catch quotas. The 
choice of scenarios is often made on an ad-hoc basis, elasticity analysis can be conducted
to identify scenarios which potentially have the biggest impact and there provide a
basis for discussing and agree which scenarios to run.

Elasticity analysis is relatively simple to apply and using life histories 
relationships allows models to be readily parameterised and case studies to be compared. This provides 
a readily available tool to researchers to pin point where uncertainty is most relevant to the sustainable 
exploitation of their particular population. 

Considering life history relationships ensures consistency in advice, allows the transfer of
knowledge about biological processes from one stock to another. It will also assist
in designing research to provide a better understanding of biological processes and
how to develop robust advice frameworks. For example, why is natural mortality of cod in the Irish Sea and bluefin tuna
in the West Atlantic assumed to be 0.2 and 0.14 respectively at all ages but for North Sea
cod and East Atlantic and Mediterranean bluefin assumed to vary with age?
What are the consequences of these assumptions and are they relatively more important than
the assumptions about spawning reproductive potential. To be consistent with life history theory $M$
should vary with age and comparative studies could help in estimating appropriate
functional relationships especially when considering exploitation at MSY.

The elasticities of SSB varied with the level of depletion and F. However, the F elasticities
did not vary. There was little difference between $F_{MSY}$ and $_{F0.1}$, but for $F_{crash}$
bigger differences were seen. In our generic simulated study, these questions were
relatively easy to answer so when applied to existing populations, elasticity analysis
will help in the choice of robust target and limit reference points to be incorporated into harvest control rules .

This study has shown that the uncertainty in some processes are more relevant to
the setting of reference points in certain situations. For the simulated population,
for both SSB and F, the natural mortality parameters $M1$ and $M2$ had the biggest
proportional effect. The next most important parameter for SSB was $a1$ (the selectivity
parameter for age at full selection). The steepness of the stock recruitment relationship
is important when considering SSB relative to $SSB_{MSY}$ and $SSB_{F0.1}$ but less
so relative to $SSB_{Fcrash}$. The other processes (growth and maturity) have similar
impacts to each other; the most important parameters are K, age at 50\% mature and age
at recruitment to the fishery. The natural mortality parameters ($M1$ and $M2$) are
again the most important process. Steepness has less of an affect compared to the
analysis for SSB. MSY has the lowest elasticites and so is the more robust reference
point for fishing mortality. Not all of these outcomes are intuitive without the analysis.
This illustrates that elasticity analysis is a potentially important method for
determining robustness of scientific advice. Since if a particular reference point
is dependent upon a parameter or process which is highly uncertain then it may be
better to find a reference point that is less dependent on that parameter or process,
especially if reducing uncertainty on that parameter is costly or difficult.
For example, if a reference points depends upon a parameter such as $M$ or steepness
that cannot be observed directly it may be better to use a reference point that is
less sensitivity to knowledge about these processes, i.e use a reference point where
uncertainty can more readily be reduced through data collection, e.g. growth and maturity.

Although we concentrated on assessing stock status relative
to reference points any system output could have been evaluated with respect to any system
parameter.

Typically, elasticity analysis is only concerned with the magnitude of the elasticity.
However, the sign or direction of the elasticity can be important when the uncertainty,
or noise, driving the parameter has an autocorrelation structure i.e. can not be
represented by white noise. For example, it has been shown that there can be important
cohort effects and autocorrelation in growth processes (REF 3 stocks paper). This
may result in several continuous years of high or low values for $K$. The direction of
the elasticity in such cases may provide enough guidance for the determination of robust reference points.
Although we considered growth, maturity, natural mortality and recruitment as separate
processes, these processes are linked. The steepness of the stock recruitment relationship
depends on spawning reproductive potential (SRP), which depends on viable egg production \cite{trippel1999estimation}
and subsequent recruitment. Recruitment is also linked with the assumptions about gonadal
growth and the processes involved in the first year of life. A single independent
estimate of $M$ is often used for the earliest life history stage i.e. eggs to the
end of the first year of life \cite{houde1989subtleties} and \cite{houde2002mortality}. 
Various mortality processes serially
affect life history stages through the first year of life, e.g. in relation to
settlement, overwintering and juvenile stages \cite{Nash2012mearly}, 
\cite{mcgurk1986natural} and  \cite{pepin1991effect}. However, there
is very little information on many of the commercially important species that will
allow an estimate of stock recruitment parameters such as steepness (e.g. \cite{mangel2010reproductive}).
The growth trajectories of individuals may not follow a von Bertalanffy growth curve
due to $M$ causing differential mortality within a cohort. Length-weight relationships
and condition can affect the maturity ogives and schedules and these can vary due to
changes in ecosystem productivity and density-dependent effects. Other factors that
need to be considered include sub-stock structure and their associated dynamics.
Examples include herring \cite{dickey2010lessons} and the influence on the assessment process \cite{kell2009lumpers}
and sub-stock structure or metapopulations are known to exist for quite a few stocks
e.g. cod in the Western \cite{frank2001contemporary} and Eastern Atlantic (North Sea) \cite{heath2008model} and bluefin tuna
in the Mediterranean \cite{rooker2007life}.
Elasticity analysis can be extended to consider such problems e.g. \cite{root1998evaluating} and \cite{henle2004role} and help identify key processes.
This study offers a simple, effective approach to not only prioritise what uncertainty
impacts the determination of reference points for exploited populations but also
provides insight in how to rationally transfer understanding between data rich
and data poor populations. It is very clear that the overall objective is not just
to reduce all uncertainty for informations sake, but to provide information for
the sustainable exploitation of populations through the setting of robust reference
points. It also highlights that uncertainty in various processes may not matter
to sustainable exploitation, whereas increasing effort to reducing uncertainty
in other processes may be of great value. The study also highlights that the ranking
of releavnt processes may change based on the exploitation status of the population.


%% == end of paper:

%% Optional Materials and Methods Section
%% The Materials and Methods section header will be added automatically.

%% Enter any subheads and the Materials and Methods text below.
\begin{materials}


\subsection{Elasticity}
Elasticity analysis is used to measure the proportional change of a system characteristic to a 
change in a system parameter. The general equation for calculating the elasticity of 
system characteristic $y$ with respect to system parameter $x$ is:

\begin{equation}
E_{y,x} = \left| \frac{\partial \ln y}{\partial \ln x} \right|        
%%     = \left| \frac{\partial  y}{\partial  x} \cdot \frac{x}{y} \right|
%%       \approx \left| \frac{ \%\bigtriangleup  y}{\%\bigtriangleup x} \right|  
%
\end{equation} 

The absolute value operator is used for simplicity although the elasticity can also be defined without the absolute value operator when the direction of 
change is important, e.g. to evaluate if a reduction in natural mortality increases or decreases MSY reference points.	

Here we calculate the elasticities of the management measures described above with respect to the life history and selectivty parameters grouped into the
categories: growth ($K$, $t_0$, $a$, $b$, $L_{\infty}$), maturity ($a50$, $ato95$ and $asym$), natural mortality ($M1$ and $M2$), the stock recruitment
relationship ($h$ and $vb$) and the selectivity ($a1$, $sl$ and $sr$). For example, the elasticity of $SSB$ relative to $SSB_{MSY}$ with respect to $L_{\infty}$ is calculated as:

\begin{equation}
E_{SSB_{MSY},L_{\infty}} = \left| \frac{\partial \ln relSSB_{MSY}}{\partial \ln L_{\infty}} \right|
\end{equation}

Elasticities are calculated for every level of $F$ used in the projections and therefore show how the current state of the stock and exploitation rate
affect the relative importance of the different life history parameters, i.e. where the most important source of uncertainty is.

Elasticities could have been calculated conditional on $L_{\infty}$ alone since paramters such as $K$, age at first maturity
and natural mortality at age can be derived from $L_{\infty}$ based on life history relationships. However, in this study these values 
were set before the elasticities are calculated, i.e. the elasticities  with respect to $L_{\infty}$ do not reflect the impact 
of $L_{\infty}$ on these life history relationships, only on the impact of the stock dynamics  through the von Bertalanffy growth and other equations.  

Mangement advice requires estimates of stock status and fishing mortality relative to target and limit reference points.
Although often aged based assessment methods are assumed to provide absolute
estimates of stock status a small change in $M$ can result in a large change in the
estimates of abundance and fishing mortality. Since $M$ is never known exactly then
advice is actually relative i.e. advice is based on whether stock or fishing mortality
is increasing or decreasing. Relative advice i.e. stock or
fishing mortality relative to a reference point has been shown to be more robust than absolute estimate \cite{kell2003evaluation}
and therefore management advice is often based on relative values.

The analysis allows use to pose questions such as are fishing mortality reference points based more robust than biomass
reference points?, are target reference points more robust than limits? is a particular reference points more robust when used as targets than a limit?
and what are the most importance biological assumptions in each case?

The elasticities in each year, (i.e. for the different levels of SSB and F) were then used to
evaluate the relative importance of the parameterisation (e.g. $K$ the rate of growth and $L_{\infty}$) 
of the various processes (i.e. growth, maturation, stock recruitment, 
natural mortality and selectivity of the fishery). 

%[Uh oh just realised we have a problem with F for fishing mortality and fecundity - maybe best to switch the F for fecundity to EP i.e. egg production or have it as Fec]

%In regard to the E for elasticity and Emigration I suggest Emigration becomes Em and Immigration becomes Im - the equation and the text needs to be fixed.


\subsection{Life History Relationships}


Empirical studies have shown that in teleosts there are significant correlations between
the life history parameters such as age at first reproduction, natural mortality, 
and growth rate \cite{roff1984evolution}. Additionally, size-spectrum theory 
and multispecies models suggest that natural mortality scales with body size 
\cite{andersen2006asymptotic}, \cite{pope2006modelling} \cite{gislason2008coexistence}. 
This means that from something that is observable, like the largest sized individuals in 
a population, it is possible to infer life history parameters for are not easily 
observed or are unavailable.

Life history characteristics and relationships between a range of stocks and species 
\cite{gislason2008coexistence} were used to parameterise an age-structured population 
model using relationships that describe growth, maturation and natural mortality.
This population was then projected at a constant fishing mortality until equilibrium 
was reached for a wide range of fishing mortalities.

The Russell equation \cite{russell1931some} summarises the key processes influencing the dynamics 
of exploited populations i.e.
 
\begin{equation}f(B_2) = B_1 + (G + R) - (F+M)\end{equation}

where biomass $B_2$ is a function of the biomass in the previous year ($B_1$), gains due to growth 
(G) and recruitment (R) and losses due to fishing (F) and natural mortality (M). Two other 
factors have been recognised since Russel originally formulated this equation: the gains through 
immigration (I) and losses through emigration (E). These modify the original equation i.e.

\begin{equation}f(B_2) = B_1 + (G+R+I) - (F+M+H)\end{equation}

Knowledge about these processes determines our ability to provide robust scientific advice. 
In this paper we concentrate  on G,R,F and M as we assume a single heterogenerous population 
with out emmigration (H) or immigration (I); however our approach could be extended to include 
H and I.

In order to provide a generic framework for modelling stock dymanics, life history relationships 
were used to parameterise appropriate functional forms for the different processes. This allows 
processes to be modelled for a range of species and stocks under a variety of assumptions.

 Growth was modelled by the Von Bertalanffy growth equation \cite{von1957quantitative}
      \begin{equation} L_t = L_{\infty}(1 - exp(-k(t-t_0)) \end{equation}
         
where $K$ is the rate at which the rate of growth in length declines as length approaches the asymptotic length $L_{\infty}$ 
and $t_{0}$ is the time at which an individual is of zero length. 

Length is converted to mass using the condition factor, $a$ and allometric growth coefficient, $b$.

\begin{equation} W = a \cdot L_t^b \end{equation}


\textbf{Recruitment} is split into Stock Reproductive Potential (SRP) and the stock recruitment relationship (SRR).

SRP is the sum of the products of the numbers of females, $n$, proportion mature-at-age, $Q$ and their mean fecundity-at-age, $G$:

   \begin{equation} SRP = \sum_{i=1}^m {n_iQ_iO_i } \end{equation}

where $m$ is the maximum age and mean fecundity-at-age is equal to 
\begin{equation} O	 = a \cdot L^{b^{\prime}} \end{equation}

If $a$ and $b$ are the same as in equation 3 then SRP is equivalent to female spawning stock biomss (SSB). However, generally the
fecundity to length relationship differs from the weight to length relationship due to variations caused by condition and age 
effects altering the relationship between weight and the number of eggs produced \cite{perez2012study}.

Proportion mature is modelled by the logistic equation with three parameters: age at 50\% ($a_{50}$) 
and 95\% ($a_{95}$) mature and the asymptotic value $m_{\infty}$. The latter allows SRP to not be equivalent 
to stock mass-at-age.

\begin{equation}
f(x) = \left\{ \begin{array}{ll}
			0                                 &\mbox{ if $(a_{50}-x)/a_{95} >  5$} \\
			a_{\infty}                        &\mbox{ if $(a_{50}-x)/a_{95} < -5$} \\
			\frac{m_{\infty}}{1.0+19.0^{(a_{50}-x)/_{95})}} &\mbox{ otherwise}
		\end{array}
       \right.
\end{equation}

%Parameters can be derived as in \cite{williams2003implications} from the theoretical relationship between $M$, $K$, and age at maturity $a_{Q}$ based on the 
%dimensionless ratio of length at maturity to asymptotic length \cite{beverton1992patterns}.

The value of $a50$ comes from the empirical relationship between $L_{\infty}$ and age at maturity \cite{gislason2008coexistence}:

\begin{equation}
  a50=0.72 logL_{\infty}^{0.93}
\end{equation}

We use a Beverton and Holt stock recruitment relationship reformulated in terms of steepness ($h$), virgin biomass ($v$) and $S/R_{F=0}$, 
where steepness is the ratio of recruitment at 20\% of virgin biomass to virgin recruitment ($R_0$). 

For the Beverton and Holt stock-recruit formulation:

\begin{equation}
R=\frac{0.8 \cdot R_0 \cdot h \cdot S}{0.2 \cdot S/R_{F=0} \cdot R_0(1-h)+(h-0.2)S}
\end{equation} 

Steepness is difficult to estimate from stock assessment data sets and there is 
often insufficient range in biomass levels that is required for its estimation 
\cite{ISSF2011steep}. Steepness and virgin biomass were set to 0.9 and 1000 t 
respectively.

Natural mortality at size is derived from the life history relationship \cite{gislason2010does}.
               
\begin{equation}
           log(M) = M1 + M2 log(L) + 1.51log(L_{\infty}) + 0.97log(k) 
\end{equation} 
where $M1$ (which determines the average natural mortality) = -2.11, $M2$ (which determines 
the rate at which natural mortality declines with length) = -1.70 and $L$ is 
the average length of the fish (in cm) for which the $M$ estimate applies. Here
we use the length at mid-year to calculate the natural mortality at age.


The model is a discrete population model where the number of individuals 
in a year-class in a year is a function of the number of individuals in 
the previous year. However, processes like growth, maturation, natural 
mortality and fishing can occur in different seasons of the year. The 
stock mass and lengths-at-age are calculated at spawning time (the start 
of year), catch mass-at-age is calculated in mid year and natural mortality 
is a function of the lengths-at-age mid year.  


\subsection{Stock projections}

Using the relationships described above we generate a fully specified 
age-structured stock based on a value of $L_{\infty}$=100 cm. The 
stock is projected forward through time at different levels of constant 
fishing pressure ranging from no fishing ($F$=0) to over exploited ($F=F_{crash}$).

Therefore, in our elasticity analysis we used relative values, i.e. SSB relative to $B_{MSY}$
and F relative to $F_{MSY}$. We also compared types of reference points, i.e. reference points designed as targets such as those
based on MSY and those designed as limits such as $F_{crash}$.

$B_{MSY}$ and $F_{MSY}$ correspond to the stock level and level of exploitation 
that provides the maximum sustainable yield.  $F_{0.1}$ is a proxy for $F_{MSY}$ and is 
the fishing mortality that corresponds to a point on the yield per recruit curve where the slope is
10\% of that at the origin. $F_{crash}$ the level of $F$ that will drive the stock to extinction.


The management measures of interest are the equilibrum SSB, yield and biomass 
relative to their reference point values corresponding to $F_{MSY}$, $F_{0.1}$ (a 
proxy for $F_{MSY}$) and $F_{crash}$ (a limit reference point), e.g. $SSB / SSB_{MSY}$, 
$SSB / SSB_{F0.1}$ etc.

SSB and $F$ relative to $F_{MSY}$, $F_{0.1}$ and $F_{crash}$ reference points were used 
as indices of stock status and exploitation. $F_{MSY}$ corresponds to the level of exploitation 
that provides the maximum sustainable yield,  $F_{0.1}$ is a proxy for $F_{MSY}$ and is 
the fishing mortality that corresponds to a point on the yield per recruit curve where the slope is
10\% of that at the origin and $F_{crash}$ the level of $F$ that will drive the stock to extinction.

In the case of $F$ equal to $F_{crash}$, SSB is 0 by definition, therefore an SSB corresponding 
to 75\% of $F_{crash}$ was used.

The calculation of reference points and fishing mortality also depend upon the selection pattern,
since not all ages are equally vulnerable to a fishery. For example, if there is a refuge for older fish,
a higher level of fishing effort will be sustainable. Also, if the fecundity of older fish is greater 
than the fecudity of younger fish of the same mass-at-age, e.g. due to maternal effects or repeat 
spawners being more fecund then a condideration of the interactions between biology and selectivity 
will be important.

Selection pattern of the fishery is represented by a double normal
(see \cite{Hilbornetal2000}) with three parameters that describe the age at 
maximum selection ($a1$), the rate at which the lefthand 
limb increases ($sl$) and the righthand limb decreases ($sr$) which 
allows flat topped or domed shaped selection patterns to be chosen.

Even in data poor situtations where catch-at-age for the entire catch time series 
is not available, some data will normally exist for some years or gears or for 
similar stocks and species. In cases where some length frequency data are available 
the shape of selection pattern, i.e. age at recruitment to the fishery, 
can be estimated using a method like that of \cite{wetherall1987estimating}.

\begin{equation}
f(x) = \left\{ \begin{array}{rl}
 2^{-[(x-a_1)/s_L]^2} &\mbox{ if $x<a_1$} \\
 2^{-[(x-a_1)/s_R]^2} &\mbox{ otherwise}
       \end{array} \right.
\end{equation}
 

\end{materials}


%% Optional Appendix or Appendices
%% \appendix Appendix text...
%% or, for appendix with title, use square brackets:
%% \appendix[Appendix Title]

\begin{acknowledgments}
Finlay Scott was funded by the UK government projects MF12-5 and MF12-1.
\end{acknowledgments}

%% PNAS does not support submission of supporting .tex files such as BibTeX.
%% Instead all references must be included in the article .tex document. 
%% If you currently use BibTeX, your bibliography is formed because the 
%% command \verb+\bibliography{}+ brings the <filename>.bbl file into your
%% .tex document. To conform to PNAS requirements, copy the reference listings
%% from your .bbl file and add them to the article .tex file, using the
%% bibliography environment described above.  

%%  Contact pnas@nas.edu if you need assistance with your
%%  bibliography.

% Sample bibliography item in PNAS format:
%% \bibitem{in-text reference} comma-separated author names up to 5,
%% for more than 5 authors use first author last name et al. (year published)
%% article title  {\it Journal Name} volume #: start page-end page.
%% ie,
% \bibitem{Neuhaus} Neuhaus J-M, Sitcher L, Meins F, Jr, Boller T (1991) 
% A short C-terminal sequence is necessary and sufficient for the
% targeting of chitinases to the plant vacuole. 
% {\it Proc Natl Acad Sci USA} 88:10362-10366.


%% Enter the largest bibliography number in the facing curly brackets
%% following \begin{thebibliography}
\bibliography{ELASTICITY} 

\begin{thebibliography}{10}

\bibitem{garcia1996precautionary}
Garcia S (1996) The precautionary approach to fisheries and its implications
  for fishery research, technology and management: an updated review.
FAO Fisheries Technical Paper : 1--76.

\bibitem{roff1984evolution}
Roff D (1984) The evolution of life history parameters in teleosts.
Canadian Journal of Fisheries and Aquatic Sciences 41: 989--1000.

\bibitem{de1986elasticity}
de~Kroon H, Plaisier A, van Groenendael J, Caswell H (1986) Elasticity: the
  relative contribution of demographic parameters to population growth rate.
Ecology 67: 1427--1431.

\bibitem{grant2003density}
Grant A, Benton T (2003) Density-dependent populations require
  density-dependent elasticity analysis: an illustration using the {LPA} model
  of tribolium {RID} c-6493-2009.
Journal of Animal Ecology 72: 94--105.

\bibitem{heppell1998application}
Heppell S (1998) Application of life-history theory and population model
  analysis to turtle conservation.
Copeia : 367--375.

\bibitem{Benton1999467}
Benton TG, Grant A (1999) Elasticity analysis as an important tool in
  evolutionary and population ecology.
Trends in Ecology and; Evolution 14: 467 - 471.

\bibitem{Hunter2000299}
Hunter C, Moller H, Fletcher D (2000) Parameter uncertainty and elasticity
  analyses of a population model: setting research priorities for shearwaters.
Ecological Modelling 134: 299 - 324.

\bibitem{Pichancourt200631}
Pichancourt JB, Burel F, Auger P (2006) A hierarchical matrix model to assess
  the impact of habitat fragmentation on population dynamics: an elasticity
  analysis.
Comptes Rendus Biologies 329: 31 - 39.

\bibitem{RogersBennett2006}
Rogers-Bennett L, Leaf RT (2006) Elasticity analyses of size-based red and
  white abalone matrix models: Management and conservation.
Ecological Applications 16: 213-224.

\bibitem{Heppell2007}
Heppell S (2007) Elasticity analysis of green sturgeon life history.
Environmental Biology of Fishes 79: 357-368.

\bibitem{kell2003evaluation}
Kell, LT and Die, DJ and Restrepo, VR and Fromentin, J.M. and Ortiz de Zarate, V. and Pallares, P. and others (2010)
  An evaluation of management strategies for Atlantic tuna stocks,
Sci. Mar. (Barc.) 2003: 353-370

\bibitem{gislason2008coexistence}
Gislason H, Pope J, Rice J, Daan N (2008) Coexistence in north sea fish
  communities: implications for growth and natural mortality.
ICES Journal of Marine Science: Journal du Conseil 65: 514--530.

\bibitem{andersen2006asymptotic}
Andersen K, Beyer J (2006) Asymptotic size determines species abundance in the
  marine size spectrum.
The American Naturalist 168: 54--61.

\bibitem{pope2006modelling}
Pope J, Rice J, Daan N, Jennings S, Gislason H (2006) Modelling an exploited
  marine fish community with 15 parameters--results from a simple size-based
  model.
ICES Journal of Marine Science: Journal du Conseil 63: 1029--1044.

\bibitem{russell1931some}
Russell E (1931) Some theoretical considerations on the overfishing problem.
Journal du conseil 6: 3.

\bibitem{von1957quantitative}
Von~Bertalanffy L (1957) Quantitative laws in metabolism and growth.
Quarterly Review of Biology : 217--231.

\bibitem{perez2012study}
P{\'e}rez-Rodr{\'\i}guez A, Morgan M, Rideout R, Dominguez-Petit R,
  Saborido-Rey F, et~al. (2012) Study of the relationship between total egg
  production, female spawning stock biomass, and recruitment of flemish cap cod
  (gadus morhua .

\bibitem{ISSF2011steep}
Anonymous (2011) Report of the 2011 issf stock assessment workshop.
ISSF Technical Report 2011-­‐02 .

\bibitem{gislason2010does}
Gislason H, Daan N, Rice J, Pope J (2010) Does natural mortality depend on
  individual size.
Fish and Fisheries 11: 149--158.

\bibitem{Hilbornetal2000}
Hilborn R, Maunder M, Parma A, Ernst B, Paynes J, et~al. (2000) Documentation
  for a general age-structured Bayesian stock assessment model: code named
  Coleraine.
FRI/UW 00/01. Fisheries Research Institute, University of Washington.

\bibitem{wetherall1987estimating}
Wetherall J, Polovina J, Ralston S (1987) Estimating growth and mortality in
  steady-state fish stocks from length-frequency data.
In: ICLARM Conf. Proc. pp. 53--74.

\bibitem{punt2008refocusing}
Punt A (2008) Refocusing stock assessment in support of policy evaluation.
Fisheries for Global Welfare and Environment : 139--152.

\bibitem{blueweiss1978relationships}
Blueweiss L, Fox H, Kudzma V, Nakashima D, Peters R, et~al. (1978)
  Relationships between body size and some life history parameters.
Oecologia 37: 257--272.

\bibitem{trippel1999estimation}
Trippel E (1999) Estimation of stock reproductive potential: history and
  challenges for canadian atlantic gadoid stock assessments.
Journal of Northwest Atlantic Fishery Science 25: 61--82.

\bibitem{houde1989subtleties}
Houde E (1989) Subtleties and episodes in the early life of fishes.
Journal of Fish Biology 35: 29--38.

\bibitem{houde2002mortality}
Houde E (2002) Mortality.
Fishery science: the unique contributions of early life stages
  Blackwell Science, Oxford : 64--87.

\bibitem{Nash2012mearly}
Nadsh R, Geffen A Mortality through the early life-history of fish: What can we
  learn from european plaice (pleuronectes platessa l.)? .

\bibitem{mcgurk1986natural}
McGurk M (1986) Natural mortality of marine pelagic fish eggs and larvae: role
  of spatial patchiness.
Marine Ecology Progress Series 34: 227--242.

\bibitem{pepin1991effect}
Pepin P (1991) Effect of temperature and size on development, mortality, and
  survival rates of the pelagic early life history stages of marine fish.
Canadian Journal of Fisheries and Aquatic Sciences 48: 503--518.

\bibitem{mangel2010reproductive}
Mangel M, Brodziak J, DiNardo G (2010) Reproductive ecology and scientific
  inference of steepness: a fundamental metric of population dynamics and
  strategic fisheries management.
Fish and Fisheries 11: 89--104.

\bibitem{dickey2010lessons}
Dickey-Collas M, Nash R, Brunel T, Van~Damme C, Marshall C, et~al. (2010)
  Lessons learned from stock collapse and recovery of north sea herring: a
  review.
ICES Journal of Marine Science: Journal du Conseil 67: 1875.

\bibitem{kell2009lumpers}
Kell L, Dickey-Collas M, Hintzen N, Nash R, Pilling G, et~al. (2009) Lumpers or
  splitters? evaluating recovery and management plans for metapopulations of
  herring.
ICES Journal of Marine Science: Journal du Conseil 66: 1776--1783.

\bibitem{frank2001contemporary}
Frank K, Brickman D (2001) Contemporary management issues confronting fisheries
  science.
Journal of Sea Research 45: 173--187.

\bibitem{heath2008model}
Heath M, Kunzlik P, Gallego A, Holmes S, Wright P (2008) A model of
  meta-population dynamics for north sea and west of scotland cod--the dynamic
  consequences of natal fidelity.
Fisheries Research 93: 92--116.



\bibitem{rooker2007life}
  Rooker, J.R. and Bremer, J.R.A. and Block, B.A. and Dewar, H. and De Metrio, G. and Corriero, A. 
   and Kraus, R.T. and Prince, E.D. and Rodriguez-Marin, E. and Secor, D.H. (2007)
  Life history and stock structure of Atlantic bluefin tuna (Thunnus thynnus).
Reviews in Fisheries Science 15:4 265--310.

\bibitem{root1998evaluating}
Root K (1998) Evaluating the effects of habitat quality, connectivity, and
  catastrophes on a threatened species.
Ecological Applications 8: 854--865.


\bibitem{henle2004role}
Henle K, Sarre S, Wiegand K (2004) The role of density regulation in extinction
  processes and population viability analysis.
Biodiversity and Conservation 13: 9--52.


\end{thebibliography}

\end{article}
%%%%%%%%%%%%%%%%%%%%%%%%%%%%%%%%%%%%%%%%%%%%%%%%%%%%%%%%%%%%%%%%

%% Adding Figure and Table References
%% Be sure to add figures and tables after \end{article}
%% and before \end{document}

%% For figures, put the caption below the illustration.
%%
%% \begin{figure}
%% \caption{Almost Sharp Front}\label{afoto}
%% \end{figure}

%% For Tables, put caption above table
%%
%% Table caption should start with a capital letter, continue with lower case
%% and not have a period at the end
%% Using @{\vrule height ?? depth ?? width0pt} in the tabular preamble will
%% keep that much space between every line in the table.

%% \begin{table}
%% \caption{Repeat length of longer allele by age of onset class}
%% \begin{tabular}{@{\vrule height 10.5pt depth4pt  width0pt}lrcccc}
%% table text
%% \end{tabular}
%% \end{table}

%% For two column figures and tables, use the following:

%% \begin{figure*}
%% \caption{Almost Sharp Front}\label{afoto}
%% \end{figure*}

%% \begin{table*}
%% \caption{Repeat length of longer allele by age of onset class}
%% \begin{tabular}{ccc}
%% table text
%% \end{tabular}
%% \end{table*}


\bf{Figure 1} Mass, natural mortality, proportion mature and selection pattern-at-age. \newline

\bf{Figure 2} Equilibrium (i.e. expected) values of SSB and yield verses fishing mortality and recruitment and yield verses SSB; points correspond to
MSY and MSY proxies ($F_{0.1}$, $F_{Max}$, SPR30\%) and limit ($F_{crash}$) reference points. \newline

\bf{Figure 3} Simulated trajectories of recruitment, SSB and yield for a increasing F. \newline

For figures with multiple panels, the first sentence of  the legend should be a brief overview of the entire figure.
\bf{Figure 4} Plots of elasticities of SSB relative to the MSY, $F_{0.1}$ and $F_{crash}$ reference points. \newline

\bf{Figure 5} Plots of elasticities of F relative to the MSY, $F_{0.1}$ and $F_{crash}$ reference points. \newline




\end{document}

