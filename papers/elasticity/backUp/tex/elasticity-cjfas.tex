%% ThzA<<Azis is file `elsarticle-template-2-harv.tex',
%%
%% Copyright 2009 Elsevier Ltd
%%
%% This file is part of the 'Elsarticle Bundle'.
%% ---------------------------------------------
%%
%% It may be distributed under the conditions of the LaTeX Project Public
%% License, either version 1.2 of this license or (at your option) any
%% later version.  The latest version of this license is in
%%    http://www.latex-project.org/lppl.txt
%% and version 1.2 or later is part of all distributions of LaTeX
%% version 1999/12/01 or later.
%%
%% The list of all files belonging to the 'Elsarticle Bundle' is
%% given in the file `manifest.txt'.
%%
%% Template article for Elsevier's document class `elsarticle'
%% with harvard style bibliographic references
%%
%% $Id: elsarticle-template-2-harv.tex 155 2009-10-08 05:35:05Z rishi $
%% $URL: http://lenova.river-valley.com/svn/elsbst/trunk/elsarticle-template-2-harv.tex $
%%
\documentclass[preprint,authoryear,12pt]{elsarticle}

%% Use the option review to obtain double line spacing
%% \documentclass[authoryear,preprint,review,12pt]{elsarticle}

%% Use the options 1p,twocolumn; 3p; 3p,twocolumn; 5p; or 5p,twocolumn
%% for a journal layout:
%% \documentclass[final,authoryear,1p,times]{elsarticle}
%% \documentclass[final,authoryear,1p,times,twocolumn]{elsarticle}
%% \documentclass[final,authoryear,3p,times]{elsarticle}
%% \documentclass[final,authoryear,3p,times,twocolumn]{elsarticle}
%% \documentclass[final,authoryear,5p,times]{elsarticle}
%% \documentclass[final,authoryear,5p,times,twocolumn]{elsarticle}

%% if you use PostScript figures in your article
%% use the graphics package for simple commands
%% \usepackage{graphics}
%% or use the graphicx package for more complicated commands
%% \usepackage{graphicx}
%% or use the epsfig package if you prefer to use the old commands
%% \usepackage{epsfig}

%% The amssymb package provides various useful mathematical symbols
\usepackage{amssymb}
%% The amsthm package provides extended theorem environments
%% \usepackage{amsthm}

%% The lineno packages adds line numbers. Start line numbering with
%% \begin{linenumbers}, end it with \end{linenumbers}. Or switch it on
%% for the whole article with \linenumbers after \end{frontmatter}.
%% \usepackage{lineno}

%% natbib.sty is loaded by default. However, natbib options can be
%% provided with \biboptions{...} command. Following options are
%% valid:

%%   round  -  round parentheses are used (default)
%%   square -  square brackets are used   [option]
%%   curly  -  curly braces are used      {option}
%%   angle  -  angle brackets are used    <option>
%%   semicolon  -  multiple citations separated by semi-colon (default)
%%   colon  - same as semicolon, an earlier confusion
%%   comma  -  separated by comma
%%   authoryear - selects author-year citations (default)
%%   numbers-  selects numerical citations
%%   super  -  numerical citations as superscripts
%%   sort   -  sorts multiple citations according to order in ref. list
%%   sort&compress   -  like sort, but also compresses numerical citations
%%   compress - compresses without sorting
%%   longnamesfirst  -  makes first citation full author list
%%
%% \biboptions{longnamesfirst,comma}

% \biboptions{}

\journal{Aquatic Living Resources}

\begin{document}

\begin{frontmatter}

%% Title, authors and addresses

%% use the tnoteref command within \title for footnotes;
%% use the tnotetext command for the associated footnote;
%% use the fnref command within \author or \address for footnotes;
%% use the fntext command for the associated footnote;
%% use the corref command within \author for corresponding author footnotes;
%% use the cortext command for the associated footnote;
%% use the ead command for the email address,
%% and the form \ead[url] for the home page:
%%
%% \title{Title\tnoteref{label1}}
%% \tnotetext[label1]{}
%% \author{Name\corref{cor1}\fnref{label2}}
%% \ead{email address}
%% \ead[url]{home page}
%% \fntext[label2]{}
%% \cortext[cor1]{}
%% \address{Address\fnref{label3}}
%% \fntext[label3]{}

\title{}

%% use optional labels to link authors explicitly to addresses:
%% \author[label1,label2]{<author name>}
%% \address[label1]{<address>}
%% \address[label2]{<address>}

\author{}

\address{}

\begin{abstract}


\end{abstract}

\begin{keyword}
%% keywords here, in the form: keyword \sep keyword

%% MSC codes here, in the form: \MSC code \sep code
%% or \MSC[2008] code \sep code (2000 is the default)

\end{keyword}

\end{frontmatter}

% \linenumbers

%% main text
\section{Introduction}
\label{Introduction}

In this study we show how life history relationships can be used to 

\begin{itemize}
 \item 
\end{itemize}

The adoption of the Precautionary Approach to fisheries management \citep{garcia1996precautionary}
requires a formal consideration of uncertainty so that risk (e.g. of stock collapse)  does not increase 
as the as the level of information decreases \citep{cooke1999improvement}. The definition of the level 
of infomation is commonly made on the availablity of catch and effort data and hence the type of stock 
assessment model that can be applied e.g. \cite{smith2008experience}. The Precautionary Approach has also meant
that more stocks, those by caught as well as targeted have to be assessed and managed, e,g, \cite{walker2011ecological}

The Precautionary Approach requires stock status to be assessed 
relative to limits and targets, to predict outcomes of management alternatives for reaching the targets 
and avoiding the limits, and to characterise uncertainty and associated risks. 
However, stock assessment data sets alone may contain insufficient information to estimate key 
parameters that determine the productivity of populations or their susceptibility to overfishing. 
This to uncertainty about biological and ecological processes such as natural mortality, 
recruitment and population structure. However, even when data are limited, empirical studies have
shown that life history parameters, such as age at first reproduction, natural mortality, and 
growth rate are correlated \citep{roff1984evolution, andersen2006asymptotic, pope2006modelling, 
gislason2008coexistence}. 

Life history relationships have been used to derive population parameters, see \citep{mcallister2001using, simon2012effects}, 
such as population growth rate (r) and the steepness ($\tau$, \cite{francis1992use})
of the stock recruitment relationship, i.e. the relative reduction in recruitment as  pawning stock biomass declines ($S$), 
Although \cite{williams2003implications} found that some biological reference points were insensitive to the choice of 
life history  parameters they were  sensitive to steepness. However, subsequently \citep{mangel2010reproductive, simon2012effects} 
showed that steepness itself is a function of life history parameters. This shows the importance of considering life history 
relationships within a common framework that explicitly considers the relationships between them.

Application of the precautionary approach to fishery management depends on the amount, type and reliability of information 
about the fishery and stocks and is applicable even with very limited information \citep{garcia1996precautionary}. A key question 
for fisheries management is therefore how can research be prioritised
so that the impact of biological uncertainty on achieving management objectives
can be reduced? In other words, how can we provide advice that is robust to uncertainty?
Answering this requires an evaluation of the relative importance of the biological assumptions 
with respect to management measures of interest. 
For example, does uncertainty about the stock recruitment relationship have a relatively bigger effect on 
yield and sustainability than uncertainty about natural mortality? 

In case of limitations of scientific knowledge  a variety of 
qualitative, semi-quantitative and quantitative modelling methods can be used (EFSA, 2013).
Important approaches in fisheries have been the  Management Strategy Evaluation (MSE e.g.  \citep{cooke1999improvement, 
mcallister2001using, kell2007flr, punt2007developing}  and Bayesian analysis e.g. \citep{haapasaari2010formalizing, uusitalo2007advantages, 
levontin2011integration, mcbride2012expert}. However, such studies can be time consuming and difficult to conduct and simpler approaches 
such as senstivity analyses can be used to evaluate the effect of changes in parameters and assumptions on management measures 
(i.e. system outputs). For example when there are unquantified uncertainties a thorough sensitivity analysis should be performed to account
for the known uncertainty and variability in all other parameter estimates \cite{hamby1994review}. This will allow the potential impact 
of the unquantified uncertaintoes on the outcome of a stock assessment to be evaluated, i.e. how different the true risk might be and how likely that is.

Another approach commonly used in financial, economic and conservation management but not fisheries is elasticity anlaysis~\cite{de1986elasticity}.
Elasticity analysis measures how changing one variable affects another. The approach can also be used within fisheries management to identify the stock assumptions 
that have the greatest impact on management measures and therefore where the impact of uncertainty may have the greatest effect on achieving managment objectives.

Elasticity measures the relative while sensitivity measures the absolute change. An elasticity analysis will tell you how if the assumptions about
the steepness of the stock recruitment relationship is more important than those about M when estimating a biological reference point such
as the maximum sustainable yield (MSY). Sensitivity analyses will tell you by how much the total allowable catch (TAC) would change.
Elasticity analysis has  proven to be a useful tool in a number of areas of population and conservation biology, for example relating changes in
vital rates to changes in the population life history~\cite{grant2003density} and to quantities of importance in management such as population
viability~\cite{heppell1998application}. Previously, elasticity anaysis has focused on terrestrial 
ecology ~\cite{Benton1999467, Hunter2000299, Pichancourt200631} with limited application to marine populations ~\cite{RogersBennett2006, Heppell2007}. 
The applicability of this approach to resource management has therefore been demonstrated and here it is used to evaluate the relative importance of
the biological assumptions made in fishery stock assessments which are too seldom questioned.


\section{Material and Methods}
\label{Methods}

\subsection{Life History}
Even when data are limited empirical studies have shown that in teleosts there is significant correlation between the life history parameters  
such as age at first  reproduction, natural mortality, and growth rate \citet{roff1984evolution}. This may mean that from something that  
is easily observable like the maximum size it is possible to infer other life history parameters, such as natural mortality that are less easy to observe.  
The biologically plausible parameter space is also restricted since size-spectrum theory and multispecies models  
suggest that natural mortality scales with body size \citet{andersen2006asymptotic}, \citet{pope2006modelling} and \citet{gislason2008coexistence}.  

Kell et al (submitted) showed how life history relationships e.g. \citet{gislason2010does} can be used to help develop simulation tools for use
in stock assessment. We extend this approach to pelagic sharks using the life history parameters for the three main shark species 
(Shortfin mako, Blue and Porbeagle) caught in Atlantic tuna fisheries.
 
Life history relationships were used to parameterise an age-structured equilibrium model, where SSB-per-recruit, yield-per-recruit  
and stock–recruitment analyses are combined, using fishing mortality ($F$), natural mortality ($M$), proportion  
mature ($Q$) and mass ($W$) -at-age with a stock–recruitment relationship. 
 
SSB-per-recruit ($S/R$) is then given by 
 
\begin{equation} 
S/R=\sum\limits_{a=r}^{n-1} {e^{\sum\limits_{i=r}^{a-1} {-F_i-M_i}}} W_a Q_a + e^{\sum\limits_{i=r}^{n-1} {-F_n-M_n}} \frac{W_n Q_n}{1-e_{-F_n-M_n}}\\ 
\end{equation}  
 
where $a$ is age, $n$ is the oldest age, and $r$ the age at recruitment. The second term is the plus-group (i.e. the summation of all ages from the  
last age to infinity).  
 
Similarily for yield per recruit ($Y/R$) 
 
\begin{equation} 
Y/R=\sum\limits_{a=r}^{n-1} {e^{\sum\limits_{i=r}^{a-1} {-F_i-M_i}}} W_a\frac{F_a}{F_a+M_a}\left(1-e^{-F_i-M_i} \right) + e^{\sum\limits_{i=r}^{n-1} {-F_n-M_n}} W_n\frac{F_n}{F_n+M_n}\\ 
\end{equation}  
 
The stock recruitment relationship can then be reparameterised so that recruitment $R$ is a function of $S/R$ 
 
e.g. for a \citet{beverton1956review}  
 
\begin{equation} 
S/R=(b+S)/a 	 
\end{equation}  
 
$S$ can then be derived from F by combining equation 3 or 4 with equation 1.  
 
There are various models to describe growth, maturation and natural mortality and the relationships between them. 
 
Here we model growth by applying \citep{von1957quantitative}   
 
\begin{equation} L_t = L_{\infty} - L_{\infty}exp(-kt) \end{equation} 
 
where $L_{\infty}$ is the asymptotic length attainable, $K$ is the rate at which the rate of growth in length declines as length approaches $L_{\infty}$, and $t_{0}$ is the time at 
which an individual is of zero length. 
 
Mass-at-age can be derived from length using a scaling exponent ($a$) and the condition factor ($b$). 
 
\begin{equation} W_t = a \times W_t^b \end{equation} 
 
 
Natural mortality ($M$) at-age can then be derived from the life history relationship \citet{gislason2008does}. 
 
\begin{equation} 
log(M) = a - b \times log(L_{\infty}) + c \times log(L) + d \times log(k) - \frac{e}{T} 
\end{equation}  
 
where $L$ is the average length of the fish (in cm) for which the M estimate applies. 
 
While maturity ($Q$) can be derived as in \citet{williams2003implications} from the theoretical relationship between M, K, and age at maturity $a_{Q}$  
based on the dimensionless ratio of length at maturity to asymptotic length \citep{beverton1992patterns}.  
 
\begin{equation} 
a_{Q}=a \times L_{\infty}-b 
\end{equation}  
 
\subsection{Stock Recruitment Relationships} 
 
Stock recruitment relationships are needed to formulate management advice, e.g. when estimating reference points such as MSY and $F_{crash}$ and making stock projections. 
Often stock recruitment relationships are reparameterised in terms of steepness and virgin biomass, where steepness  
is the ratio of recruitment at 40\% of virgin biomass to recruitment at virgin biomass. However, steepness is difficult to estimate from  
stock assessment data sets: there is often insufficient range in biomass levels to allow the estimation of steepness \citet{ISSF2011steep}. 
 
We use a Beverton and Holt stock recruitment relationship reformulated in terms of steepness ($h$), virgin biomass ($v$) and $S/R_{F=0}$. 
 
Where steepness is the proportion of the expected recruitment produced at 20\% of virgin biomass relative to virgin recruitment $(R_0)$. For the Beverton \& Holt  
stock-recruit formulation, this equals 
 
\begin{equation} 
R=\frac{0.8 \times R_0 \times h \times S}{0.2 \times S/R_{F=0} \times R_0(1-h)+(h-0.2)S} 
\end{equation}  
 
For future studies however, it may be more appropraite to alter the stock recruitment relationship used here. For some species, 
particularly those with low  
fecundity, a more appropriate stock–recruitment relationship may  
be one that is expressed in terms of offspring survival rather 
than recruitment. Unlike fish producing millions of eggs, species 
with low fecundity (e.g. sharks), produce few offspring per litter  
and exhibit relatively little variability in litter size among spawn-  
ers. This suggests both low productivity in general and a more  
direct connection between spawning output (which is commonly 
expressed in numbers of eggs or embryos) and recruitment than  
for many species. The commonly used Beverton–Holt and Ricker 
models can be stated in terms of pre-recruit survival, with two  
parameters controlling the shape of the function. Both models, however, would result in survival decreasing fastest at low stock 
size (concave decreasing survival) even though it is  reasonable 
to expect that for low fecundity species, offspring survival would 
instead decrease faster due to competition when the population 
approaches carrying capacity (convex decreasing survival). New methods such as the a , flexible three-parameter 
stock–recruitment model (Maunder–Taylor-Methot stock recruitment 
model), based on pre-recruit survival should be investigated. This new model enables 
the description of a wider range of pre-recruit survival curves than 
either Beverton–Holt or Ricker, including those that correspond to 
shapes ranging from convex to concave \citet{taylor2012srr}. 

\subsection{Observer Programmes}

\subsection{Power Analysis}
A power analysis is conducted to determine the ability to detect trends in abundance for different life histories, population distributions
and sampling levels.
.
The power analysis is conducted for a range of survey CVs based on different observer sampling levels and evaluates the number of years 
required before an given change in the population can be detected. The power of a change abundance being detected is 
calculated using linear regression given i) estimates of survey variability (CV), ii) the number of annual surveys, iii) 
the relationship between CV and population density and iv) the percent rate of change (see Gerrodette, 1987 and 1991).
 
Conducting a power analysis requires choosing appropriate power and significance levels. The power of a statistical test is 
the probability of correctly rejecting a null hypothesis ($H_0$) when the hypothese is false (in this case the $H_0$ is that 
there has been no increase in the population).  As the power increases, the chances of a Type II error (i.e. a false negative) 
occurring decrease (Greene 2000). Conventionally a test with power greater than 0.8 level (or $\beta<=.2$) is considered statistically 
powerful.

The statistical power determines the ability of a test to detect an effect, if the effect actually exists (High 2000). The significance level is chosen depending on the acceptable risk of drawing the wrong conclusion. Smaller levels of $\alpha$ increase confidence in the determination of significance, but run an increased risk of failing to reject a false null hypothesis (a Type II error, or "false negative"), and so have less statistical power. The selection of the level $\alpha$ thus inevitably involves a compromise between significance and power, and consequently between the Type I error and the Type II error.

It was also assumed i) that the survey CV was independent of population size (i.e. consistent with the stock assessment assumptions) and ii) that the population increase was linear (since the stock is recovering to $B_{MSY}$ and so density dependence will limit population increase). 

Table X show the the population increase required to detect a significant upward trend (at the 95\% level) with a power of 80\%, while figure Y shows the spawning stock biomass (SSB) for the six projection trajectories used to provide management advice to the Commission by the SCRS (values are relative to the 2012 level). While table Z summaries the CVs by area for the different survey designs.

These table and figures allow important questions to be answered for example e.g.
If stock is a single stock what CV will be required to detect a doubling of the population within 10yrs?

Table X shows that if the CV is 25\% then it will take 11 years whilst if the CV is 20\% it will take 6 years, so the answer would be a CV of 20-25\%. The next question would be what survey design would provide a CV of between 20% & 25%. From table Y it can be seen that.

The CV of the survey is a factor that can be controlled to some extent by the design of and the funding for a survey. The population increase is determined by the biology of a stock and managment. Even with perfect managment as assumed by the Commission and the SCRS there is however considereable uncertainty about the response of the stock to managment, i.e. the SCRS projections predict that stock may increase between 50% and 200% by 2023.


All modelling was conducted in R using the /pkg{fishmodels} and /pkg{FLR} packages

\section{Discussion}	
A fuller consideration of uncertainty within fisheries advice frameworks can be
performed used Bayesian approaches or Management Strategy Evaluation (MSE). MSE is
commonly used to evaluate the impact of different management measures, given a broad
range of uncertainty. However, performing an MSE is a costly process in terms of human
resources and can take several years. Therefore, tools such as elasticity analysis,
which is comparatively less demanding to carry out, are important to help identify
and focus research and management efforts. For example, is it more important to
reduce uncertainty about the stock recruitment relationship or natural mortality
or to develop harvest control rules that are robust to such uncertainty? Elasticity
analyses can be used to answer such questions and prioritise research effort. It
can also shift the current focus from defining populations either as data poor or
rich defined solely on fishery catch and effort towards a better understanding of
biological processes. Here we demonstrate the use of elasticity analysis for
prioritising research effort with a study of the population dynamics of a fish population
based on life history theory. As such the study is not modelled on one species of
fish. First we simulate a population based on life history relationships [REF] and
then by projecting the population from an unfished to an over-exploited state.
We then calculate elasticities to allow us to evaluate the relative importance of
the different system or biological parameters when assessing the population relative
to system characteristics defined by biological reference points. This allows us
to address two important questions: what is the relative importance of the different
biological processes in providing advice and and how robust is advice based on the
common biological reference points?


%% The Appendices part is started with the command \appendix;
%% appendix sections are then done as normal sections
%% \appendix

%% \section{}
%% \label{}

%% References
%%
%% Following citation commands can be used in the body text:
%%
%%  \citet{key}  ==>>  Jones et al. (1990)
%%  \citep{key}  ==>>  (Jones et al., 1990)
%%
%% Multiple citations as normal:
%% \citep{key1,key2}         ==>> (Jones et al., 1990; Smith, 1989)
%%                            or  (Jones et al., 1990, 1991)
%%                            or  (Jones et al., 1990a,b)
%% \citep{key} is the equivalent of \citet{key} in author-year mode
%%
%% Full author lists may be forced with \citet* or \citep*, e.g.
%%   \citep*{key}            ==>> (Jones, Baker, and Williams, 1990)
%%
%% Optional notes as:
%%   \citep[chap. 2]{key}    ==>> (Jones et al., 1990, chap. 2)
%%   \citep[e.g.,][]{key}    ==>> (e.g., Jones et al., 1990)
%%   \citep[see][pg. 34]{key}==>> (see Jones et al., 1990, pg. 34)
%%  (Note: in standard LaTeX, only one note is allowed, after the ref.
%%   Here, one note is like the standard, two make pre- and post-notes.)
%%
%%   \citealt{key}          ==>> Jones et al. 1990
%%   \citealt*{key}         ==>> Jones, Baker, and Williams 1990
%%   \citealp{key}          ==>> Jones et al., 1990
%%   \citealp*{key}         ==>> Jones, Baker, and Williams, 1990
%%
%% Additional citation possibilities
%%   \citeauthor{key}       ==>> Jones et al.
%%   \citeauthor*{key}      ==>> Jones, Baker, and Williams
%%   \citeyear{key}         ==>> 1990
%%   \citeyearpar{key}      ==>> (1990)
%%   \citetext{priv. comm.} ==>> (priv. comm.)
%%   \citenum{key}          ==>> 11 [non-superscripted]
%% Note: full author lists depends on whether the bib style supports them;
%%       if not, the abbreviated list is printed even when full requested.
%%
%% For names like della Robbia at the start of a sentence, use
%%   \Citet{dRob98}         ==>> Della Robbia (1998)
%%   \Citep{dRob98}         ==>> (Della Robbia, 1998)
%%   \Citeauthor{dRob98}    ==>> Della Robbia


%% References with bibTeX database:

\bibliography{refs.bib}
\bibliographystyle{abbrvnat}

European Food Safety Authority. Guidance on the environmental risk assessment of genetically modified animals. EFSA Journal 2013;11(5):3200


%% Authors are advised to submit their bibtex database files. They are
%% requested to list a bibtex style file in the manuscript if they do
%% not want to use model2-names.bst.

%% References without bibTeX database:

% \begin{thebibliography}{00}

%% \bibitem must have one of the following forms:
%%   \bibitem[Jones et al.(1990)]{key}...
%%   \bibitem[Jones et al.(1990)Jones, Baker, and Williams]{key}...
%%   \bibitem[Jones et al., 1990]{key}...
%%   \bibitem[\protect\citeauthoryear{Jones, Baker, and Williams}{Jones
%%       et al.}{1990}]{key}...
%%   \bibitem[\protect\citeauthoryear{Jones et al.}{1990}]{key}...
%%   \bibitem[\protect\astroncite{Jones et al.}{1990}]{key}...
%%   \bibitem[\protect\citename{Jones et al., }1990]{key}...
%%   \harvarditem[Jones et al.]{Jones, Baker, and Williams}{1990}{key}...
%%

% \bibitem[ ()]{}

% \end{thebibliography}

\end{document}

Fisheries management is concerned with trying to set, and then achieve, realistic
management objectives \cite{cochrane2000reconciling}. This is carried out through defining
management objects and correspoding measures of interest, for example spawning stock biomass 
and fishing mortality (F), and then setting values of a range of reference points in order to achieve 
management objectives. This is formalised in ht

McBride and Burgman in Expert Knowledge and Its Application in Landscape Ecology
hen there is somewhat less Finnish link to Bayesian methodology and participatory modelling which sort of pertains to what we are doing in Marine
Policy "The added value of partipatory modelling in fisheries management - what has been learnt?" By Rockmann et al
Edwards, C.T.T, Hillary, R.M., Levontin, P., Blanchard, J.L., and K. Lorenzen. (2012). Fisheries assessment and management:
a synthesis of common approaches with special reference to deepwater and data-poor stocks. Reviews in fisheries science. 20(3):136-153


%%
%% End of file `elsarticle-template-2-harv.tex'.
