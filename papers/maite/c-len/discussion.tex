Bias and precision are both important factors to consider when assessing fish stocks, bias reflects how close an estimate is to an accepted value and precision reflects reproducibility of the estimate. For example, if an assessment is to be re-conducted every year to monitor the impact of a management measure, a precise but biased method would be able to detect a trend better than an unbiased but imprecise method. As with scientific instruments this trade-offtradeoff require calibration, which in the case of fish stock assessment can be performed using MSE, where the choice of parameters and reference points in a management procedure are tuned, i.e. calibrated, to meet the desired management objectives as represented by the OM. Therefore, a biased method (e.g., DBSRA) may be preferable to one that is less biased, but more imprecise (e.g., LIME). Alternatively, imprecision can be addressed through the choice of the percentile (e.g., median being the 50\% percentile value) for the derived model output used by management (e.g., catch or SPR); assuming that the true value is contained within the parameter distribution For example, instead of taking the median value, one could instead use the derived model output associated with the 40th percentile. Such an approach (Ralston et al. 2011) is used in fisheries management systems to directly incorporate scientific uncertainty (both bias and imprecision), and can also be tuned using MSE. 