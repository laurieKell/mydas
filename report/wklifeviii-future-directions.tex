
Trends and fluctuations in populations are determined by complex interactions between extrinsic forcing and intrinsic dynamics. For example, stochastic recruitment can induce low‐frequency variability, i.e. ‘cohort resonance’, which can induce apparent trends in abundance and may be common in age‐structured populations; such low‐frequency fluctuations can potentially mimic or cloak critical variation in abundance linked to environmental change, over‐exploitation or other types of anthropogenic forcing (Bjørnstad, 2004). Although important, these effects can be difficult to disentangle. The simulations so far show that life histories are important and should be used to help condition operating models to ensure robust feedback-control rules. MSE is important to help develop these robust feedback control rules and to help identify appropriate observational systems.
Although the performance of the HCR depended on the life-history characteristic, it was not in the way initially expected, i.e. the outcomes could not be grouped solely by whether the Operating Models (OMs) represented fast growing vs. late maturing species or demersal vs. pelagic stocks. What was important was the nature of the dynamics, i.e. how variable was the stock between years; for example, a stock could exhibit high interannual variability if natural mortality and recruitment variability was high, regardless of the values of k, Linf, L50. The nature of the indices is also important; for example, even if a stock had low interannual variability, an index could be highly variable if it was based on juveniles or there were large changes in spatial distribution between years. It is therefore necessary to look at the robustness of HCRs to the nature of the time-series of the stock (as represented by the OM) and to the characteristics of the data collected from it (as represented by the Observation Error Model). This will require tuning by constructing a reference set of OMs and then tuning the HCR to secure the desired trade-offs. The work so far can be considered as focusing first on developing HCR that perform satisfactorily for a reference set, the next step is to develop case-specific HCRs.


\begin{enumerate}[8]

\item Aspects to consider for the 3.2.1 rule by the next meeting would be:

\begin{enumerate}[8.1]

\item Investigating the impact of relative weighting of the r, f and b components of the rule on the performance of the rule;

\item Investigating more extensively the time-lag properties of the r component, including alternative formulations;

\item Setting of appropriate reference levels in the f and b component of the rules, and the extent to which this could be done with tuning that depends on life-history traits and/or the nature of the time-series;

\item Investigation of the use of trends in an index without a reference level.

\end{enumerate}

\item Longer term aspect to consider for data-limited rules:

\begin{enumerate}[9.1]

\item Focusing on the nature of time-series and developing diagnostics that could help determine the rules that would work well under alternative characterisations of the nature of the time-series, and aspects such as quality of data used by the rules (and hence ability to detect signals), ability to set appropriate reference points, etc.;

\item Linking life-history traits, the form of density-dependence and fishery characteristics (e.g. including fishery selectivity) to the nature of resulting time-series;

\item Develop guidance for use of catch rules by linking (a) and (b);

\item Avoiding the shot-gun approach to simulation testing e.g. by making more extensive use of sensitivity (elasticity) analysis to highlight factors that are most important in determining the time-series behaviour of stocks;

\item Investigating the implications of how the operating models are set up (fishing history, depletion levels, selectivity assumptions, mortality) on the behaviour of the stock and on the performance of the catch rule.

\end{enumerate}

